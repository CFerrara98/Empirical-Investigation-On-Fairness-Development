\phantomsection
%\addcontentsline{toc}{chapter}{Introduzione}
\chapter{Introduzione}
\markboth{Introduzione}{}
% [titolo ridotto se non ci dovesse stare] {titolo completo}

\section{Motivazioni e Obiettivi} %\label{1sec:scopo}
Al giorno d'oggi la costante e continua crescita applicativa dei sistemi di intelligenza artificiale, nei più svariati ambiti professionali, ha indotto l'intera comunità scientifica nell'ambito IT a porsi nuovi quesiti ed obiettivi che permettano la realizzazione di soluzioni dagli alti standard qualitativi ed immuni a numerose tipologie di vulnerabilità. Negli ultimi anni in particolar modo, si osserva come le soluzioni AI-Intensive, ed in particolar modo i moduli di machine learning, necessitino della formalizzazione di standard di sviluppo che tengano conto delle specifiche di sviluppo e delle caratteristiche intrinseche (e.g. la forte correlazione tra dati di training e l'algoritmo utilizzato) che differenziano in termini strutturali questa particolare tipologia di soluzioni. In particolare, questa tipologia di mancanze, ha portato ad un progressivo e costante interesse ed avvicinamento dei ricercatori nell'ambito dell'ingegneria del software alle problematiche dei moduli AI-Intensive \cite{rech2004artificial} ed alla nascita di una nuova branca di studio che appunto accomuna i processi ingegneristici al contesto specifico, ovvero l'ingegneria del software per l'intelligenza artificiale. Dalla progettazione dei primi processi di analisi, progettazione e sviluppo dei moduli AI-Intesive, l'ingegneria del software ha portato alla nascita di nuovi standard di sviluppo veri e propri che caratterizzano in modo specifico l'intero ciclo di vita di una soluzione AI-Intensive, tra i più innovativi e famosi, rientra sicuramente l'approccio evolutivo di standard basati su pipeline quali MLOps \cite{burkov2020machine}. Un aspetto chiave di questa particolare tipologia di standard di sviluppo è senz'altro l'analisi intrinseca e continua dei differenti aspetti qualitativi che un modulo ML intensive deve rispettare, e proprio tra gli oramai noti e standardizzati aspetti di qualità, e.g. Sicurezza, Accuracy, Performances etc. \cite{NFRForML}, il mondo della ricerca osserva come negli ultimi anni i moduli di machine learning, siano sempre più soggetti a nuove tipologie di vulnerabilità, che portano il modulo stesso all'operare in maniera eticamente scorretta, con conseguenti risultati discriminatori per particolare tipologie di utenti appartenenti a particolari gruppi di utenti, meno o mal rappresentati rispetto l'intero gruppo di addestramento, i così detti gruppi minoritari \cite{brun2018software}. \\

Quello che si evidenzia dai primi studi nell'ambito è che l'aspetto fairness di un modulo di machine learning è un concetto difficilmente definibile e misurabile in modo univoco data la stretta correlazione con l'ambiente applicativo delle specifica soluzione \cite{FairnessDefinitionExplained}. Osservando con occhi critico lo stato dell'arte, si osserva come molti studi, abbiano portato notevoli migliorie verso il comune obiettivo di progettare e realizzare soluzioni ML-Intensive eticamente corrette, da strategie specifiche per l'analisi dei requisiti e delle migliorie dei dati fino alle più specifiche strategie di testing ed analisi, d'altra parte però è facilmente riscontrabile come le specifiche strategie progettate e realizzate dal mondo della ricerca, abbiano ad oggi poco riscontro con quelle che sono le pratiche adottate o le esigenze specifiche nel trattamento di fairness di chi professionalmente realizza soluzioni ml-intensive in ambito aziendale. Tenendo conto di tale mancanza, questo lavoro di tesi, a mezzo di un Survey su larga scala, si pone quindi l'obiettivo di investigare a 360° su quali siano i punti di forza e debolezza del trattamento di fairness in ambito lavorativo all'attuale stato della pratica, in modo tale da fornire, non solo una visione strutturata dei plausibili spunti di ricerca futuri per la tematica specifica, ma anche la consapevolezza che in ambito professionale la tematica dell'etica di un modulo ML-Intensive, sia qualcosa di estremamente concreto, che necessita di standard e tecniche di sviluppo adeguate al pari di altre caratteristiche qualitative che appunto rendono un modulo ML-Intensive adatto ad essere applicato in un contesto d'uso reale.

\section{Risultati}
Tramite l'analisi empirica condotta, quello che questo lavoro di tesi ha effettivamente osservato è che allo stato della pratica, il concetto di software fairness necessiti ancora di un mirato supporto della ricerca al fine di maturare in modo opportuno al fine di essere trattato sistematicamente al pari di altre specifiche non funzionali di un modulo di machine learning, in particolare si evidenzia come:

\begin{itemize}
    \item Nonostante l'ampio catalogo di definizioni ed approcci noti per il trattamento alla fairness, il trattamento formale dei requisiti etici di una soluzione intelligente sia attualmente poco standardizzato, in particolare si notano visioni poco concordanti riguardo gli approcci noti in letteratura e l'utilizzo di altre strategie specifiche non totalmente coperte dagli attuali studi;
    \item Fairness sia ancora un requisito poco maturo per essere considerato effettivamente primario rispetto ad altre specifiche più note quali accuracy o sicurezza, viene anche osservato come questo principio sia variabile a seconda dei specifici domini applicativi;
    \item La progettazione di strategie specifiche o il riutilizzo di quanto già a disposizione in letteratura nel trattamento di fairness, sia effettivamente necessario e utile, soprattutto adoperando modelli di sviluppo evolutivi, e.g. MLOps;
    \item La suddivisione di responsabilità e il coinvolgimento di esperti possa essere un fattore estremamente rilevante nel trattamento di fairness in ciascuna delle fasi di sviluppo;
    \item Le aziende siano consapevoli che non è più prescindere dal trattamento e risoluzione di vulnerabilità discriminatorie nelle soluzioni prodotte, infatti è osservabile come la maggior parte degli esperti coinvolti ritengano che le proprie aziende trattino le problematiche etiche a differenti livelli di maturità. In generale si osserva come gli esperti si collochino massivamente a livelli basilari della scala proposta, ma osservando il campione nella sua interezza, è intuibile come nei prossimi anni ci possa essere una notevole tendenza al miglioramento. 
\end{itemize}
 
 Ovviamente lo studio condotto, non fornisce risultati definitivi per trattare le problematiche etiche in maniera definitivamente corretta, ma data la natura ad ampio raggio dell'analisi, si cerca appunto di lasciare punti di osservazione che possano essere utili per future analisi che tengano conto sempre di più di ciò che i professionisti del settore necessitino o ritengano utile.
\section{Struttura della tesi}

Il documento di tesi è strutturato nel seguente modo:

\begin{itemize}
    \item \textbf{Capitolo 2 - Background}: Il capitolo di background riporta una visione generale dei macro-ambiti in cui il lavoro di tesi si colloca, fornendo una panoramica ad alto livello dei concetti basilari di ingegneria del software ed intelligenza artificiale, fino ai concetti specifici di ingegneri del software per l'intelligenza artificiale direttamente richiamati nei capitoli successivi;
    \item \textbf{Capitolo 3 - Stato dell'Arte}: Il capitolo stato dell'arte, fornisce dettagli di ricerca sulla tematica cardine del lavoro di tesi ovvero la Software Fairness nei modelli di intelligenza artificiale, si forniscono dettagli su definizioni ed esempi noti per il trattamento di fairness che hanno impattato la ricerca negli ultimi anni, per poi concludere con una panoramica di alcuni studi specifici (related works) inerenti la software fairness nell'ambito dell'intelligenza artificiale e dell'ingegneria del software;
    \item \textbf{Capitolo 4 - Design}: Il capitolo di design e progettazione, fornisce dettagli su quelli che sono gli obiettivi di ricerca del lavoro di indagine empirica, per poi passare ai dettagli di progettazione e diffusione del Survey sottomesso ai lavoratori dell'ambito;
    \item \textbf{Capitolo 5 - Analisi}: Il capitolo di analisi, raccoglie tutte le procedure utili all'elaborazione dei dati ottenuti a seguito della chiusura del Survey, passando dalle metodologie di data cleaning e pre-processing effettuate sul dataset originale, per poi passare alla vera e propria analisi dei dati e generalizzazione dei risultati con relativi punti di discussione e implicazioni derivanti;
   
    \item \textbf{Capitolo 6 - Conclusioni}
    
     \item \textbf{Bibliografia}
\end{itemize}
