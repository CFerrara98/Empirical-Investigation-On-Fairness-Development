\chapter{Analisi e validazione dei dati} %\label{1cap:spinta_laterale}
% [titolo ridotto se non ci dovesse stare] {titolo completo}
%
     \section{Data Cleaning}
    
    Come osservato precedentemente, il survey Prolific, ha raccolto un totale di 203 risposte totali. Delle quali, 19 sono state considerate inattendibili e quindi rimosse dal dateset di partenza per i seguenti motivi:
    
    \begin{itemize}
        \item 17 sono state considerate inattendibili a causa di risposte errate alle domande poste come attenction check di verifica, quindi successivamente eliminate dal dataset originale.
        \item 2 risposte sono state eliminate, osservando inconguenze con l'identificativo Prolific immesso.
    \end{itemize}
     
    Delle 184 risposte considerate come utili ai fini dell'analisi, c'è inoltre da considerare che 68 partecipanti hanno dichiarato di non aver alcuna esperienza con lo sviluppo di moduli AI-intensive, quindi sono stati direttamente condotti alla sezione di chiusura del Survey elettronico. I restanti 116 partecipanti all'indagine hanno invece dichiarato di avere effettivamente esperienza con lo sviluppo di soluzioni AI-Intensive, quindi le risposte relative saranno quindi utilizzate al fine di elaborare i risultati dello studio in risposta agli obiettivi di ricerca prefissati precedentemente.  Da notare inoltre come 29 partecipanti all'indagine considerati come fonte attendibile (con o senza esperienza nello sviluppo di soluzioni AI-Intensive), abbiano lasciato a disposizione un loro contatto email per eventuali interviste future, o per ottenere nuovi aggiornamenti circa l'elaborazione dei risultati.  
    
    \section{Pre-processing}
    \section{Analisi dei dati}
    \section{Minacce alla validità}

\newpage
