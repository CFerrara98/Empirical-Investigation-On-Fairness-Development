\chapter{Conclusioni} %\label{1cap:spinta_laterale}
% [titolo ridotto se non ci dovesse stare] {titolo completo}
%


	In questo lavoro di tesi, si è osservato come la progettazione e lo sviluppo aziendale di un canonico modulo di machine learning, non possa più prescindere dall'analizzare problematiche di carattere etico. Si è osservato inizialmente, tramite un'attenta analisi dello stato dell'arte, che l'aspetto etico, meglio conosciuto con la sua traduzione fairness, di un modulo di machine learning  è notoriamente considerato in letteratura come un aspetto variegato e difficile da generalizzare, e per questo la comunità scientifica, sia nell'ambito dell'intelligenza artificiale, che dell'ingegneria del software, ha orientato gli studi di ricerca al fine di progettare e testare soluzioni che possano essere di supporto a chi sviluppa soluzioni intelligenti di trattare l'etica di un modulo ML-Intensive in modo più accurato e concreto. Volendo quindi verificare se e come i progressi della ricerca fossero percepiti allo stato della pratica aziendale, si è deciso di condurre un'indagine empirica, a mezzo di un Survey su larga scala, che appunto coinvolgesse gli esperti del settore per rispondere in maniera puntuale e il più completa possibile ad un principale obiettivo di ricerca, per l'appunto: \emph{In che modo il concetto di Software Fairness è attualmente percepito nell'ambito lavorativo ML-Intensive?}. Specializzando quindi l'obiettivo principale in 5 obiettivi specifici che riguardassero appunto:
	\begin{itemize}
	    \item Definizioni ed approcci pratici per il trattamento della fairness;
	    \item Composizione ottimale di un canonico team di sviluppo ML-Intensive Fair Critical;
	    \item Analisi di rilevanza dei livelli di Fairness rispetto altre specifiche non funzionali;
	    \item Applicabilità di strategie specifiche per il trattamento di Fairness in una canonica pipeline ML;
	    \item Livello di maturità aziendale nel trattamento della fairness;
	\end{itemize}

    si è passati quindi alla disseminazione su larga scala, che in pochi giorni ha prodotto 203 risposte, poi analizzate e ridotte a 116 prima (per controlli ed operazioni di data cleaning) di passare poi all'analisi dei dati ed alla conseguente generalizzazione dei risultati.\\\\
    
    Dai dati osservati si è principalmente osservato che allo stato attuale della  pratica, i professionisti concordano principalmente come il concetto di moduli di machine learning eticamente corretti, sia altamente dipendenti dal dominio di applicazione, e che l'una o l'altra definizione dipenda effettivamente dal contesto applicativo. In particolare poi, dai dati a disposizione, si evince come Fairness sia un concetto che necessita di ulteriori studi, prima di essere considerato a tutti gli effetti un aspetto di qualità maturo al pari di altre specifiche non funzionali. In particolare l'analisi dei dati evidenzia come:
    
    \begin{itemize}
        \item L'applicazione di strategie o approcci vada effettivamente sistematizzata ed a seconda del dominio d'utilizzo, ma contestualmente sia necessario comprendere bene la rilevanza di ciascun approccio nel contesto d'uso per sfruttarne a pieno le potenzialità;
        \item Il trattamento della fairness nello sviluppo ML-Intensive necessiti effettivamente di un intensiva collaborazione tra Data Scientists ed Ingegneri del Software, e capire in che fasi di una pipeline siano critiche queste figure è attualmente ancora un punto da sistematizzare;
        \item Durante lo sviluppo sia critico coinvolgere manager di progetto o \emph{esperti} nel trattamento di fairness per garantire che le specifiche etiche siano rispettate, anche in questo caso sarà necessario approfondire bene le responsabilità o le competenze delle due figure;
        \item Fairness necessiti di strategie specifiche durante l'intero ciclo di vita di un modulo ML-Intensive, e se si prende come riferimento gli specifici moduli di machine learning, con particolare riferimento alle fasi evolutive del modello ovvero preparazione dei dati e delle feature, analisi e validazione del modello e reporting di statistica con il modello in esecuzione.
    \end{itemize}


    Dall'analisi ad ampio raggio condotta, quello che si evince quindi che il concetto di Fairness in un modulo ML-intensive, è un qualcosa da sistematizzare e approfondire con studi mirati che tengano conto effettivamente di quanto già sia stato fatto e di cosa è necessario osservando con occhio critico le pratiche aziendali. In particolare si osserva come possano essere particolarmente critici studi che:
    
    \begin{itemize}
        \item Analizzino nello specifico quali siano le principali cause di discriminazioni o le opportune pratiche da adottare o meno nello sviluppo sviluppo Pipeline oriented, in modo tale da definire dei veri e propri protocolli di sviluppo fair-specific come già si fa con altre specifiche non funzionali;
        \item Guidino l'apprendimento delle nozioni basilari di fairness, in ambito accademico o aziendale, tenendo conto della stretta correlazione con il dominio applicativo, magari applicando definizioni o approcci a casi di studio noti, ed analizzandone vantaggi e svantaggi;
        \item Aiutino, di conseguenza, i professionisti nel discriminare a seconda dei casi quali approcci adoperare, in questo caso potrebbe essere anche utili approfondire l'applicabilità di altri metodi emersi dalle risposte ottenute come l'analisi condotta tramite metodologie empiriche o l'utilizzo di indicatori statistici di correlazione in fase di preparazione dei dati;
        \item Definiscano protocolli gestionali e di composizione del team sistematici che guidino lo sviluppo ML-Intensive Fair-Critical, definendo per l'appunto ruoli e responsabilità specifiche in ogni fase di sviluppo.
    \end{itemize}
    
    Va infine osservato, come l'attuale studio ribadisca che quanto stato già fatto sia effettivamente utile al fine degli obiettivi di lungo raggio che la ricerca si pone. Infatti, è facilmente osservabile che strategie specifiche analizzate a priori dell'investigazione empirica, abbiano fatto si che le aziende incrementassero notevolmente il loro grado di consapevolezza in merito il fattore etico delle soluzioni progettate. Quello che effettivamente si cerca di suggerire è che tramite il coinvolgimento di esperti professionali, oltre la progettazione di nuove strategie mirate, sarà anche possibile sistematizzare e applicare concretamente quanto già è stato proposto dalla ricerca, con il fine ultimo di rendere i sistemi di machine learning ampiamente robusti rispetto qualsiasi tipo di problematica di carattere etico/discriminatorio.
\newpage
