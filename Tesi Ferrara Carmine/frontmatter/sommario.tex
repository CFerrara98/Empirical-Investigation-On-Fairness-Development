%\selectlanguage{italian}

\begin{center}
    \LARGE \textsc{Abstract}
\end{center}


L'applicazione di soluzioni di Machine Learning è sempre più utilizzata per rispondere alla più ampia varietà di problemi in contesti d'utilizzo reali, tra i più noti aspetti qualitativi di Accuracy, Performances, Sicurezza, etc. che caratterizzano un modulo di machine learning, è sempre più rilevante valutarne le caratteristiche etiche ed intervenire in maniera adeguata qualora si manifestino problematiche discriminatorie. Negli ultimi anni all'interno della comunità scientifica, in particolare nell'ambito dell'Ingegneria del Software per l'Intelligenza Artificiale, si sta riscontrando una  notevole crescita dell'aspetto fairness come tematica di ricerca, tuttavia ciò che ancora risulta ancora essere mancante, è proprio tener conto di come le pratiche lavorative di chi si approccia quotidianamente allo sviluppo di soluzioni ML intensive fair-critical, possano agevolare e migliorare le stesse attività di ricerca. Tenendo conto di questa mancanza, questo lavoro di tesi, si è posto l'obiettivo principale di proporre alla ricerca una visione empirica di quali siano gli aspetti caratterizzanti del trattamento di fairness in ambito lavorativo, cercando contestualmente di sensibilizzare gli esperti del settore all'adottare sempre più processi di sviluppo fair-oriented standardizzati. Analizzando i risultati dello studio, ottenuti a mezzo di un Survey su larga scala, si osserva come la Fairness anche allo stato della pratica sia un aspetto strettamente dipendente dal dominio applicativo e strettamente dipendenti dallo stesso restano sicuramente validi definizione e approcci alla misurazione dei livelli di fairness noti in letteratura. Tuttavia viene anche evidenziato l'utilizzo approcci tecnico pratici molto interessanti, quali l'applicazione di metodologie empiriche, tecniche di ottimizzazione dei dati, o l'utilizzo di indicatori statistici di correlazione. Per gli esperti del settore, fairness continua ad essere ad oggi un aspetto meno standardizzato rispetto ad altre specifiche non funzionali, considerando però come il livello di maturità nel trattamento di fairness stia progressivamente aumentando. Le aziende stesse suggeriscono come praticamente fairness sia vista come un aspetto che evolve e migliora durante il ciclo di vita di un modulo di machine learning, evidenziando come strategie fair-oriented specifiche che impattino sull'evoluzione dei dati raccolti, e/o la validazione del modello siano cruciali nel trattamento di Fairness in approcci di sviluppo basati su pipeline quali MLOps, tra l'altro viene evidenziata la cruciale importanza di figure professionali come Manager o Esperti di Etica per il trattamento di fairness durante lo sviluppo di un modulo di machine learning. Da quanto emerso è possibile inoltre costatare, come la stretta correlazione tra dominio applicativo e fairness di un modulo di machine learning è probabilmente anche un'ottima chiave di lettura per la didattica accademica.

