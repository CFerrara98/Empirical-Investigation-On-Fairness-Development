%\selectlanguage{italian}
\begin{abstract}
%\chapter*{Abstract}
Oggigiorno, è sempre più importante osservare come applicativi della società moderna richiedano soluzioni di intelligenza artificiale per la più variegata tipologia di problemi, alcuni degli esempi più noti sono sistemi di diagnostica per le malattie, sistemi bancari per la concessione dei prestiti, ma ne esistono tanti altri. Al fine di operare in maniera opportuna nel settore di utilizzo, queste soluzioni necessitano che tutta una serie di prerogative non funzionali vengano soddisfatte, tra le più importanti vi è senz'altro considerare l'abilità di lavorare in maniera eticamente corretta. In altri termini, come suggerisce la ricerca è necessario che un sistema AI-Intensive che si rispetti debba essere necessariamente Fair-Aware, cioè libero da bias di tipo etico che ne influenzino l'operato per alcune tipologie minoritarie di utenti. Il mondo della ricerca, in particolar modo le comunità di ricerca inerenti l'intelligenza artificiale e l'ingegneria del software, ha studiato il problema della Fairness in maniera più approfondita, osservando che il concetto è difficilmente generalizzabile e dipende molto dalle caratteristiche degli specifici problemi. In particolare, molte sono le definizioni di Fairness emerse dai differenti studi e di varia natura sono le metriche ad esse correlate. Ma effettivamente, fin dove queste problematiche sono considerate nella pratica? Come  esperti, quali data scientist o ingegneri del software si approcciano a questo problema? In che modo la Fairness viene considerata durante lo sviluppo? E soprattutto in quali fasi del ciclo di vita di una soluzione AI-Intensive viene considerata?. Obiettivo principale di questo lavoro di tesi è, quindi, quello di interrogare gli esperti del settore, al fine di fornire un'idea più precisa di come il concetto di Software Fairness sia recepito nella pratica, identificare quali sono le vulnerabilità maggiori che possono rendere un tool AI eticalmente scorretto e soprattutto capire e formalizzare quali possono essere le best-practice da adottare quando il concetto di Software Fairness è una delle prerogative fondamentali da rispettare durante l'intero ciclo di vita di un sistema AI-Intensive.
\end{abstract} 